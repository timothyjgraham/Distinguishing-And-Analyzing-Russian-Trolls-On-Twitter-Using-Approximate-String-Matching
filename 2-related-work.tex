%!TEX root = main.tex

\section{Background and related work}
We structure prior work into two parts.
\autoref{sec:tardeANT} briefly introduces the monad theoretical framework for measuring social roles using social media data, 
and \autoref{subsec:political-trolls-detection} presents some related work on social media trolls and bots. 
%; and (3) work relating to edit distance and approximate string matching.

\subsection{Quantifying social roles using online traces}
\label{sec:tardeANT}

\textbf{A new (old) view of society}. 
Gabriel Tarde's ancient theory of monadology~\citep{tarde2011monadology} has recently been adapted into the body of social theory known as Actor-Network Theory (ANT). 
It promises a powerful framework for the study of identity and social change in heterogeneous networks \citep{latour2012whole}. 
In the 19$^{th}$ century, Tarde's ideas proved not only difficult to conceptualise but even more difficult to operationalise due to a lack of data. It is perhaps for this reason that his alternative approach to describing social processes was not empirically testable and subsequently relegated to a footnote in history.

However, \citet{latour2012whole} argue that the onset of the information age and the availability of digital data sets make it possible to revisit Tarde’s ideas and render them operational. 
By examining the digital traces left behind by actors in a network (human and non-human), Latour et al. (2012: 598) argue that we can `slowly learn about what an entity ``is'' by adding more and more items to its profile'. 
The radical conclusion is that datasets `allow entities to be individualised by the never-ending list of particulars that make them up' (Latour et al. 2012: 600). 
Hence, a \textit{monad} is a `point of view, or, more exactly, a type of navigation that composes an entity through other entities' (Latour et al. 2012: 600).

As an example of this form of analysis, \citet{latour2012whole} use the example of looking up an academic named `Herve C.' on the web to show how collecting information through various digital sources results in the assemblage of a network that defines an actor's identity. 
As the authors argue: `The set of attributes - the network - may now be grasped as an envelope - the actor - that encapsulates its content in one shorthand notation' (Latour et al. 2012: 593). 
Instead of atomic nodes (micro) that somehow `enter into' or `end up forming' structures (macro), we have a very different view of identity: in order to know what something is and understand its role in society, we simply follow the traces that it leaves behind through \textit{its relations to other entities}, or in other words we trace its monad:

%% MAR: if in need of space, this quote can go away. It is nice, but not necessary.
%% Dongwoo: i agree. the only problem is that we refer back to this quote in the final discussion section.
\begin{quote}
If for instance we look on the web for the curriculum vitae of a scholar we have never heard of before, we will stumble on a list of items that are at first vague. Let's say that we have been just told that `Herve C.' is now `professor of economics at Paris School of Management'. At the start of the search it is nothing more than a proper name. Then, we learn that he has a `PhD from Penn University', `has written on voting patterns among corporate stake holders',`has demonstrated a theorem on the irrationality of aggregation', etc. If we go on through the list of attributes, the definition will expand until paradoxically it will narrow down to a more and more particular instance. Very quickly, just as in the kid game of Q and A, we will zero in on one name and one name only, for the unique solution: `Herve C.'. Who is this actor? Answer: this network. What was at first a meaningless string of words with no content, a mere dot, now possesses a content, an interior, that is, a network summarised by one now fully specified proper name (Latour et al., 2012: 592).
\end{quote}

\subsection{Political trolls on social media}
\label{subsec:political-trolls-detection}

\textbf{Bots and trolls in political discussions.}
In recent years online trolls and social bots have attracted considerable scholarly attention. 
Online trolls tend to be either human, `sock puppets' controlled by humans \citep{Kumar-et-al}, or semi-automated accounts that provoke and draw people into arguments or simply occupy their attention \citep{herring2002searching} for amplify particular messages and manipulate discussions \citep{broniatowski-et-al,ferrara-et-al-trolls-ideology}.
Recent studies have investigated the impact of trolls and bots in social media to influence political discussions \citep{FM7090}, spread fake news \citep{shao2017spread}, and affect the finance and stock market \citep{ferrara2016rise}.
Especially, in a political context, studies have shown that online trolls mobilised support for Donald Trump's 2016 U.S. Presidential campaign \citep{ICWSM1613006}, and, of particular interest to this paper, were weaponised as tools of foreign interference by Russia during and after the 2016 U.S. election \citep{boatwrighttroll,2018who-let-the-trolls-out}. 
%% MAR: not sure we want to be so definitive "it is evident". Soothing a bit the message
It is the current understanding that Russian trolls successfully amplified a largely pro-Trump, conservative political agenda during the 2016 U.S. Election, and managed to attract both bots and predominantly conservative human Twitter users as `spreaders' of their content \citep{ferrara-et-al-trolls-ideology,stewart2018examining}. 
%%Save space a bit
%Notably, analysis of Russian troll retweet networks suggests that they tried to further accentuate disagreement and foster division in politically polarised conversations surrounding race and gun violence \citep{stewart2018examining}. 

%While trolls and bots have become increasingly prevalent and influential, methods to detect and analyse them in social networks has also received wide attention \citep{cook2014twitter}. For social bots, the state-of-the-art method for detecting them on Twitter is the BotOrNot API \citep{davis2016botornot}. BotOrNot uses a Random Forest classifier, an ensemble supervised learning method, to measure the likelihood of a user being a bot based on more than 1000 features extracted from meta-data, patterns of activity, and tweet content \citep{varol2017online}, and can achieve a performance of 0.95 AUC (Area Under ROC Curve). Additionally, \cite{morstatter2016new} propose a bot detection model which considers recall in the formulation and achieve a better performance than the heuristics and topic modelling baseline methods. \cite{ferrara2016rise} also mention many other detection methods such as detection based on social network information and based on crowd-sourcing and leveraging human intelligence. However, all of these methods requires extensive information about each account including their network structures. It is also unclear whether the methods can take into account topical changes in the social media space. %However, all of these methods require the costly annotation of training data, and requires many attributes such as friends, followers, content and sentiment of tweet, social network patterns, and activity time series \citep{varol2017online}. 

%On the other hand, automatically distinguishing trolls from regular users is a difficult problem, as there is currently no unified definition of what they are across multiple contexts online, and current approaches rely on manually annotated training data sets \citep{mihaylov2015finding} that are costly in terms of time and resources. Moreover, in addition to the general troll, there are also \textit{different types} of trolls. For example, sockpuppet accounts can take on different roles in online discussions, such as supporting vs non-supporting vs dissenting to sway sentiment and/or create the illusion of group consensus in comment threads \citep{Kumar-et-al}. Similarly, different types of trolls may be employed by specific individuals or groups to achieve specialised goals, such as trolls employed by the Internet Research Agency (IRA) to influence the political discourse and public sentiment in the United States \citep{boatwrighttroll}. The Clemson researchers \cite{boatwrighttroll} used advanced tracking software of social media to collect tweets from a large number of accounts that Twitter has acknowledged as being related with the IRA. The researchers first qualitatively analyse the Twitter data then use quantitative analysis to explore how behaviour changes over time.

\textbf{Detection and role of trolls.}
While trolls and bots have become increasingly prevalent and influential, methods to detect and analyse their role in social networks has also received wide attention \citep{cook2014twitter,davis2016botornot,varol2017online}.
On the other hand, identifying and differentiating specific sub-groups or types of trolls poses a difficult challenge, which has attracted relatively less attention.
%For example, sockpuppet troll accounts can take on different roles in online discussions, such as supporting vs non-supporting vs dissenting to sway sentiment and/or create the illusion of group consensus in comment threads \citep{Kumar-et-al}. Similarly, 
Different types of trolls can be employed by specific individuals or groups to achieve specialised goals, such as trolls employed by the Internet Research Agency (IRA) to influence the political discourse and public sentiment in the United States \citep{boatwrighttroll,stewart2018examining}. 
%\textbf{Tracking trolls.}
The Clemson researchers \cite{boatwrighttroll} used advanced tracking software of social media to collect tweets from a large number of accounts that Twitter has acknowledged as being related with the IRA. Using qualitative methods, the researchers identified five types of trolls, namely right troll, left troll, news feed, hashtag gamer, and fearmonger \citep{boatwrighttroll}. They found that each type of troll exhibited vastly different behaviour in terms of tweet content, reacted differently to external events, and had different patterns of activity frequency and volume over time \citep[p.10-11]{boatwrighttroll}. 
%Although the Russian troll handles generally behaved consistently according to their type, there were instances where handles switched categories, such as the fearmonger trolls. Trolls of different types also worked in unison to advance the agenda of the IRA. 
%To make sense of the troll types and to explain their role and motives with respect to the broader Russian political agenda, the Clemson researchers closely analysed the content of tweets, in particular the hashtags that were deployed, and the timing of the tweets in relation to external events that occurred during the period of analysis. For example, right troll activity had a massive spike in response to the Chelsea bombing on September 17th, 2016 \citep[p.11]{boatwrighttroll}.

\textbf{Relation to our work.}
Therefore we observe a close connection between \textit{semantics} (what trolls talk about), \textit{temporality} (when they are active and how hashtags are deployed over time), and the particular \textit{roles} and strategies that drive their behaviour (e.g. right versus left troll). 
Our interest in this paper is to develop and evaluate a framework that can accurately identify roles of users within a population (in this case Russian troll types) by clustering them based not only on semantics but also temporality, that is, the order in which activities occur over time. To our knowledge this has not been achieved in previous work, where temporal information is often ignored or disregarded in analysis, such as the use of cosine distance and Levenshtein edit distance to differentiate sockpuppet types \citep{Kumar-et-al}, network-based methods to infer the political ideology of troll accounts \citep{ferrara-et-al-trolls-ideology}, or word embeddings and hashtag networks with cosine distance as edge weights \citep{2018who-let-the-trolls-out}. Where temporal analysis of troll activity has been undertaken, the focus has been on tweet volume and hashtag frequency at different time points \citep{2018who-let-the-trolls-out} rather than how roles and strategies change over time. % However, before proceeding further we discuss related work in social theory, which provides an important theoretical and conceptual basis to anchor the problem we address in this paper.