%!TEX root = main.tex

%\secmoveup
\section{Russian trolls dataset}
\label{sec:troll-dataset}

This study uses a publicly available dataset of verified Russian troll activity on Twitter, published by Clemson University researchers \citep{boatwrighttroll}\footnote{Available at https://github.com/fivethirtyeight/russian-troll-tweets/}. The complete Russian troll tweets dataset consists of nearly 3 million tweets from 2,848 Twitter handles associated with the Internet Research Agency, a Russian ``troll factory''. It is considered to be the most comprehensive empirical record of Russian troll activity on social media to date. 
The tweets in this dataset were posted between February 2012 and May 2018, most between 2015 and 2017.

In this work, we aim to distinguish trolls based solely on their authored text.
We focus on the tweet content, and we try to detect the author category for each tweet (i.e. what type of Russian troll it was authored by), rather than detect the author category for each handle. 
There are multiple types of Russian trolls categorised by the Clemson researchers, namely: right troll; news feed; left troll; hashtag gamer; and fearmonger, sorted in decreasing order by tweet frequency. 
In this research, we will focus on the top 3 most frequent trolls: right troll, news feed and left troll. In addition, in this study we only consider English-language tweets, although future work can easily generalise our methods to any language expressed as Unicode. 
This amounts to a subset of 1,737,210 tweets emitted by 733 accounts.

According to \citet{boatwrighttroll}, right trolls behave like ``MAGA\footnote{MAGA is an acronym that stands for Make America Great Again. It was the election slogan used by Donald Trump during his election campaign in 2016, and has subsequently become a central theme of his presidency.} 
Americans'', they mimic typical Trump supporters and they are highly political in their tweet activity. 
On the other hand, left trolls characteristically attempt to divide the Democratic Party against itself and contribute to lower voter turnout. 
They achieve this by posing as mimic Black Lives Matter activists, expressing support for Bernie Sanders\footnote{Bernie Sanders was the alternative Democrat Presidential Nominee}, and acting derisively towards Hillary Clinton. 
While tweets posted by left and right trolls have a strong political inclination, news feeds trolls tend to present themselves as legitimate local news aggregators with the goal of contributing to, and magnifying, public panic and disorder.

%\todo[inline]{DW: I guess we can qualitatively verify the claims made in the last couple of paragraph about tactics between different types of trolls using the visualisation. So we may tone down a bit here and say it was widely perceived that the trolls were working in these ways? and we could show that this is actually somewhat happened in our dataset in section 5.3.}